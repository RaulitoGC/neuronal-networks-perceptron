\documentclass[12pt]{article}
\usepackage[utf8]{inputenc}
\usepackage{titlesec}
\usepackage{url}

\titleclass{\subsubsubsection}{straight}[\subsection]
\newcounter{subsubsubsection}[subsubsection]
\renewcommand\thesubsubsubsection{\thesubsubsection.\arabic{subsubsubsection}}
\renewcommand\theparagraph{\thesubsubsubsection.\arabic{paragraph}}
\titleformat{\subsubsubsection}
  {\normalfont\normalsize\bfseries}{\thesubsubsubsection}{1em}{}
\titlespacing*{\subsubsubsection}
{0pt}{3.25ex plus 1ex minus .2ex}{1.5ex plus .2ex}

\makeatletter

\def\toclevel@subsubsubsection{4}
\def\l@subsubsubsection{\@dottedtocline{4}{7em}{4em}}

\makeatother

\setcounter{secnumdepth}{4}
\setcounter{tocdepth}{4}


\titleclass{\subsubsubsubsection}{straight}[\subsection]
\newcounter{subsubsubsubsection}[subsubsubsection]
\renewcommand\thesubsubsubsubsection{\thesubsubsubsection.\arabic{subsubsubsubsection}}
\renewcommand\theparagraph{\thesubsubsubsubsection.\arabic{paragraph}}
\titleformat{\subsubsubsubsection}
  {\normalfont\normalsize\bfseries}{\thesubsubsubsubsection}{1em}{}
\titlespacing*{\subsubsubsubsection}
{0pt}{3.25ex plus 1ex minus .2ex}{1.5ex plus .2ex}

\makeatletter

\def\toclevel@subsubsubsubsection{5}
\def\l@subsubsubsubsection{\@dottedtocline{5}{11em}{4em}}

\makeatother

\setcounter{secnumdepth}{5}
\setcounter{tocdepth}{5}

\title{}

\begin{document}

\tableofcontents

\section{Introducción}

El cerebro humano es un sistema de cálculo muy complejo, puede llevar a cabo procesamientos que a primera vista parecen sencillos, como por ejemplo, el reconocimiento de imágenes. Esta capacidad que tiene el cerebro humano para pensar, recordar y resolver problemas ha inspirado a muchos científicos a intentar imitar estos funcionamientos.\hfill \break

Los intentos de crear un ordenador que sea capaz de emular estas capacidades ha dado como resultado la aparición de las llamadas Redes Neuronales Artificales o Computación Neuronal.\hfill \break

El principal objetivo del Reconocimiento de patrones es la clasificación ya sea supervisada o no supervisada. Aplicaciones como Data Mining, Web Searching, recuperación de datos multimedia, reconocimiento de rostros, reconocimientos de caracteres escritos a mano, etc., requieren de técnicas de reconocimiento de patrones robustas y eficientes.
Las redes neuronales, por su capacidad de generalización de la información disponible y su tolerancia al ruido, constituyen una herramienta muy útil en la resolución de este tipo de problemas.\cite{patterRecognition}\hfill \break

Este documento muestra los conceptos básicos de las Redes Neuronales y su regla de aprendizaje, en particular la configuración en \textit{Perceptrón Multicapa} y el varios algoritmos de aprendizaje (Propagación hacia atrás,Métodos de segundo orden, RPROP, Algoritmos Genéticos).


\clearpage

\section{Neurona Artificial}
asd

\subsection{Redes Neuronales}
asd

\subsubsection{Redes Neuronales Supervisadas}
asd
\subsubsubsection{Redes Neuronal Perceptron Multicapa}

asd
\subsubsection{Redes Neuronales No Supervisadas}
asd


\section{Caso de Estudio}
asd
\subsection{Problema}
asd
\subsection{Justificacion}
asd
\subsection{Propuestas de Solucion}
asd
\subsubsection{Topologia}
asd
\subsubsection{Reglas}
asd
\subsubsubsection{Regla de propagación}
asd
\subsubsubsection{Regla de Activacion}
asd
\subsubsubsection{Regla de Salida}
asd
\subsubsubsection{Regla de Aprendizaje}
asd
\subsubsubsubsection{Back Propagation}
asd
\subsubsubsubsection{Segundo Orden}
asd
\subsubsubsubsection{R-PROP}
asd
\subsubsubsubsection{Algoritmos Geneticos}
asd

\subsection{Desarrollo de la solucion}
asd
\subsubsection{Herramientas}
asd
\subsubsubsection{R}
asd
\subsubsubsection{RStudio}
asd
\subsubsubsection{Package}
asd
\subsubsection{Implementacion}
asd
\subsubsubsection{Back Propagation}
asd
\subsubsubsection{Segundo Orden}
asd
\subsubsubsection{R-PROP}
asd
\subsubsubsection{Algoritmos Geneticos}
asd

\subsection{Resultados}
asd
\subsection{Conclusiones}
asd




\bibliographystyle{unsrt}
\bibliography{bibliography}

\end{document}
